\documentclass[10pt]{article}
\textwidth = 450pt
\headsep = 2pt
\headheight = 1pt
\oddsidemargin = 1pt

\usepackage{changepage}
\usepackage{fancyvrb}
\usepackage{graphicx}
\usepackage{amsmath}
\usepackage{capt-of}
\usepackage{amsfonts}
\usepackage{verbatim}

%%%%%%%%%%%%%%%%%%% Code %%%%%%%%%%%%%%%%%%%%%%%
\usepackage{color}
\usepackage[table]{xcolor} %adding background color to your tables
\usepackage{listings}% Allows you to present C++ syntax as it looks
\usepackage{listings} %enables inputing code set the settings below
\definecolor{dkgreen}{rgb}{0,0.45,0}
\definecolor{gray}{rgb}{0.2,0.5,0.5}
\definecolor{mauve}{rgb}{0.58,0,0.82}
%\definecolor{purple}{RGB}}{204, 45, 109}
\lstset{ %
language=C, % choose the language of the code
commentstyle=\color{dkgreen},
basicstyle=\footnotesize, % the size of the fonts that are used for the code
numbers=left, % where to put the line-numbers
numberstyle=\footnotesize, % the size of the fonts that are used for the line-numbers
stepnumber=1, % the step between two line-numbers. If it is 1 each line will be numbered
numbersep=5pt, % how far the line-numbers are from the code
backgroundcolor=\color{white}, % choose the background color. You must add \usepackage{color}
showspaces=false, % show spaces adding particular underscores
showstringspaces=false, % underline spaces within strings
showtabs=false, % show tabs within strings adding particular underscores
frame=single, % adds a frame around the code
tabsize=2, % sets default tabsize to 2 spaces
captionpos=b, % sets the caption-position to bottom
breaklines=true, % sets automatic line breaking
breakatwhitespace=false, % sets if automatic breaks should only happen at whitespace
keywordstyle=\color{purple}, % keyword style
numberstyle=\tiny\color{gray}, % the style that is used for the line-numbers
rulecolor=\color{black}, % if not set, the frame-color may be changed on
stringstyle=\color{blue}, % string literal style
escapeinside={\%*}{*)} % if you want to add a comment within your code
}
\DefineVerbatimEnvironment{code}{Verbatim}{fontsize=\small}
\DefineVerbatimEnvironment{example}{Verbatim}{fontsize=\small}
%%%%%%%%%%%%%%%%%%%%%%%%%%%%%%%%%%%%%%%%%%%%%%%%

\begin{document}
\title{Using Kinect to Evaluate Dance Performances\\ Third Year Group Project}
\author{Stylianos Venieris, Marcin Baginski, Theo Pavlakou, Zeping Xue, \\ Yijie Ge \& Hesam Ipakchi  }
\date{\today}
\maketitle
\newpage

\section*{\center Abstract}

\section{Kinect \& NiTE Software Evaluation}
\noindent
To determine a method of evaluating the dance student, the limitations of the camera and the NiTE software must first be evaluated with respect to the criteria addressed below. To do this we use the UserViewer Application that comes as a sample with the NiTE software library with the camera elevated 75cm above the ground, within the 60cm to 180cm range that is suggested by Microsoft for optimal tracking. 

\subsection{Camera Range}
\noindent 
To test for the camera's range, we lay a tape measure on the ground starting from directly below the camera up to 8m away from the camera. We then use two subjects of different heights and body shapes to evaluate the performance of the camera and the software for tracking at different distances. The subject first starts within a few centimetres of the camera and slowly moves backwards until the camera calibrates and starts tracking and continues to do so until the tracking is lost. After this, the subject is required to start from the depths of the room, much further than the range of the camera, and to start walking slowly towards the camera, again taking a record of the following specifications. The results can be shown in Tables \ref{cam_range_180_away} to \ref{cam_range_150_toward}.
\\
\begin{table}[h]
\center
\begin{tabular}{ | l | c |}
\hline
Distance from Camera/cm & Description of Performance \\
\hline
60 & Identification of subject. Tracking. No skeleton.\\
120 & Skeleton fitted\\
410 & Tracking is lost\\
\hline
\end{tabular}
\caption{Subject moving away from camera. Subject height 180cm.}
\label{cam_range_180_away}
\end{table}

\begin{table}[h]
\center
\begin{tabular}{ | l | c |}
\hline
Distance from Camera/cm & Description of Performance \\
\hline
100 & Identification of subject. Tracking. No skeleton.\\
120 & Skeleton fitted\\
410 & Tracking is lost\\
\hline
\end{tabular}
\caption{Subject moving towards camera. Subject height 180cm.}
\label{cam_range_180_toward}
\end{table}

\begin{table}[h]
\center
\begin{tabular}{ | l | c |}
\hline
Distance from Camera/cm & Description of Performance \\
\hline
70 & Identification of subject. Tracking. No skeleton.\\
110 & Skeleton fitted\\
430 & Tracking is lost\\
\hline
\end{tabular}
\caption{Subject moving away camera. Subject height 150cm.}
\label{cam_range_150_away}
\end{table}

\begin{table}[h]
\center
\begin{tabular}{ | l | c |}
\hline
Distance from Camera/cm & Description of Performance \\
\hline
50 & Identification of subject. Tracking. No skeleton.\\
110 & Skeleton fitted\\
430 & Tracking is lost\\
\hline
\end{tabular}
\caption{Subject moving towards camera. Subject height 150cm.}
\label{cam_range_150_toward}
\end{table}

\subsection{Effect of Varying Lighting Conditions}
\subsection{Obstruction in Range}
\subsection{Velocity of Movement}
\subsection{Camera Angle Relative to Subject}
\subsection{Multiple People}
\subsection{Multiple Cameras}
\clearpage

\section*{Appendix}
\subsection*{Some Code}
\begin{lstlisting}
int main()
{
	// your code
	x = 5;
}

\end{lstlisting}
\end{document}
