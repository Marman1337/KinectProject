\documentclass[10pt]{article}
\textwidth = 450pt
\headsep = 2pt
\headheight = 1pt
\oddsidemargin = 1pt

\usepackage{changepage}
\usepackage{fancyvrb}
\usepackage{graphicx}
\usepackage{amsmath}
\usepackage{capt-of}
\usepackage{amsfonts}
\usepackage{verbatim}

%%%%%%%%%%%%%%%%%%% Code %%%%%%%%%%%%%%%%%%%%%%
\usepackage{color}
\usepackage[table]{xcolor} %adding background color to your tables
\usepackage{listings}% Allows you to present C++ syntax as it looks
\usepackage{listings} %enables inputing code set the settings below
\definecolor{dkgreen}{rgb}{0,0.45,0}
\definecolor{gray}{rgb}{0.2,0.5,0.5}
\definecolor{mauve}{rgb}{0.58,0,0.82}
%\definecolor{purple}{RGB}}{204, 45, 109}
\lstset{ %
language=C, % choose the language of the code
commentstyle=\color{dkgreen},
basicstyle=\footnotesize, % the size of the fonts that are used for the code
numbers=left, % where to put the line-numbers
numberstyle=\footnotesize, % the size of the fonts that are used for the line-numbers
stepnumber=1, % the step between two line-numbers. If it is 1 each line will be numbered
numbersep=5pt, % how far the line-numbers are from the code
backgroundcolor=\color{white}, % choose the background color. You must add \usepackage{color}
showspaces=false, % show spaces adding particular underscores
showstringspaces=false, % underline spaces within strings
showtabs=false, % show tabs within strings adding particular underscores
frame=single, % adds a frame around the code
tabsize=2, % sets default tabsize to 2 spaces
captionpos=b, % sets the caption-position to bottom
breaklines=true, % sets automatic line breaking
breakatwhitespace=false, % sets if automatic breaks should only happen at whitespace
keywordstyle=\color{purple}, % keyword style
numberstyle=\tiny\color{gray}, % the style that is used for the line-numbers
rulecolor=\color{black}, % if not set, the frame-color may be changed on
stringstyle=\color{blue},
escapeinside={\%*}{*)} % if you want to add a comment within your code
}
\DefineVerbatimEnvironment{code}{Verbatim}{fontsize=\small}
\DefineVerbatimEnvironment{example}{Verbatim}{fontsize=\small}
%%%%%%%%%%%%%%%%%%%%%%%%%%%%%%%%%%%%%%%%%%%%%%%%

\begin{document}
\title{Using Kinect to Evaluate Dance Performances\\ Third Year Group Project}
\author{Stylianos Venieris, Marcin Baginski, Theo Pavlakou, \\Zeping Xue, Yijie Ge \& Hesam Ipakchi  }
\date{\today}
\maketitle
\newpage

\section*{\center Abstract}

\section{Kinect \& NiTE Software Evaluation}
\noindent
To determine a method of evaluating the dance student, the limitations of the camera and the NiTE software must first be evaluated with respect to the criteria addressed below. To do this we use the UserViewer Application that comes as a sample with the NiTE software library with the camera elevated 75 cm above the ground, within the 60 cm to 180 cm range that is suggested by Microsoft for optimal tracking. 

\subsection{Camera Range}
\noindent 
To test for the camera's range, we lay a tape measure on the ground starting from directly below the camera up to 8 m away from the camera. We then use two subjects of different heights and body shapes to evaluate the performance of the camera and the software for tracking at different distances. The subject first starts within a few centimetres of the camera and slowly moves backwards until the camera calibrates and starts tracking and continues to do so until the tracking is lost. After this, the subject is required to start from the depths of the room, much further than the range of the camera, and to start walking slowly towards the camera, again taking a record of the following specifications. The results can be shown in Tables \ref{cam_range_180_away} to \ref{cam_range_150_toward}.
\\
\begin{table}[h]
\center
\begin{tabular}{ | l | c |}
\hline
Distance from Camera/cm & Description of Performance \\
\hline
60 & Identification of subject. Tracking. No skeleton.\\
120 & Skeleton fitted\\
410 & Tracking is lost\\
\hline
\end{tabular}
\caption{Subject moving away from camera. Subject height 180 cm.}
\label{cam_range_180_away}
\end{table}

\begin{table}[h]
\center
\begin{tabular}{ | l | c |}
\hline
Distance from Camera/cm & Description of Performance \\
\hline
100 & Identification of subject. Tracking. No skeleton.\\
120 & Skeleton fitted\\
410 & Tracking is lost\\
\hline
\end{tabular}
\caption{Subject moving towards camera. Subject height 180 cm.}
\label{cam_range_180_toward}
\end{table}

\begin{table}[h]
\center
\begin{tabular}{ | l | c |}
\hline
Distance from Camera/cm & Description of Performance \\
\hline
70 & Identification of subject. Tracking. No skeleton.\\
110 & Skeleton fitted\\
430 & Tracking is lost\\
\hline
\end{tabular}
\caption{Subject moving away camera. Subject height 150 cm.}
\label{cam_range_150_away}
\end{table}

\begin{table}[h]
\center
\begin{tabular}{ | l | c |}
\hline
Distance from Camera/cm & Description of Performance \\
\hline
50 & Identification of subject. Tracking. No skeleton.\\
110 & Skeleton fitted\\
430 & Tracking is lost\\
\hline
\end{tabular}
\caption{Subject moving towards camera. Subject height 150cm.}
\label{cam_range_150_toward}
\end{table}
\noindent
These results highlight the fact that the camera has a restricted field of view and therefore the number of people that each camera can support is limited. This will have to be further considered when finalising a design for the application. One thing to note is that, since  the optimum distance away from the camera, with respect to tracking, is in the range of 1.5m to 3.5m it would not be beneficial to have the camera(s) in the corners of the room. This would cause a huge amount of the already small range to be wasted due to the height of the room. 

\subsection{Effect of Varying Lighting Conditions}
\noindent
In order to evaulate the effect of varying lighting conditions on the tracking capabilities of the camera, we utilise a lux meter to determine the intensity of light in a room. The meter has been calibrated such that a complete darkness represents 0 lux. We then perform a series of tests to determine, if the varying lighting conditions have an effect on the tracking range of the camera. The table below presents the results of the tests:
\begin{table}[h]
\center
\begin{tabular}{ | l | c |}
\hline
Light intensity & Tracking range (starts - stops tracking) \\
\hline
4 lux & 120 cm - 420 cm\\
16 lux & 120 cm - 420 cm\\
44 lux & 120 cm - 420 cm\\
142 lux & 120 cm - 420 cm\\
176 lux & 120 cm - 420 cm\\
\hline
\end{tabular}
\caption{Tracking range in varying light intensity of the room}
\label{cam_range_varying_light}
\end{table}

\noindent
Since the skeleton is primarily fitted using the depth sensor of the RGB-D camera, the light intensity was expected to have little, if any, effect on the tracking range of the camera. This has been confirmed in the experiments.


\subsection{Obstruction in Range}
\noindent
To test for the possibility that obstructions could interfere with the fitting of the skeleton on the subject, we place a chair in various positions around the subject and in front of the subject. We then do the same with another person walking in the vicinity of the subject. This scenario is probably more relevant to the situations which may be encountered in a dance class room.  
\subsubsection{Chair}
\noindent Positioning the chair adjacent to the subject produces interesting results. When the chair is directly in front of the subject, the NiTE software tries to fit a skeleton to the chair as well as the person. This can be seen in Figure \textbf{ADD FIGURE}. Putting the chair on the head of the subject and at the waist does not give this result, but instead the proportion of the skeleton that is created by the software is quite distorted (see Figure \textbf{ADD FIGURE}). 
\subsubsection{Person}
\noindent
In this experiment the second person walks within different ranges of the camera and the initial subject. Interestingly enough, even at quite fast speeds of movement, the camera can calibrate and track both people quite quickly. This will most probably not be a limitation when creating the software. 
\subsection{Velocity of Movement}


\subsection{Camera Angle Relative to Subject}


\subsection{Multiple People}


\subsection{Multiple Cameras}
\clearpage

\section*{Appendix}
\subsection*{Some Code}
\begin{lstlisting}
int main()
{
	// your code
	x = 5;
}

\end{lstlisting}
\end{document}
